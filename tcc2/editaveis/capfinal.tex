\chapter[Resultados Obtidos]{Resultados Obtidos}

O ciclo menstrual tem duas fases principais: a fase folicular e a fase lútea. Dessas, a fase folicular 
pode 
ser dividida em fase folicular inicial, na qual ocorre a menstruação, e a fase folícular final, na qual 
há a liberação do óvulo.
Quando o óvulo é liberado, tem início a fase lútea, que também pode ser dividida em fase lútea inicial e 
fase lútea final, sendo quando, para algumas pessoas, acontece a TPM.

Durante essas fases, as pessoas passam por várias mudanças hormonais, que podem influenciar nas tarefas 
cotidianas, demandando mais energia para a realização de certas tarefas. Na fase folicular final e lútea inicial,
foi relatada uma melhora significativa no desempenho das atividades cognitivas e verbais, enquanto no final da fase 
lútea, há a relação com o aumento do cortisol, hormônio do stress e uma certa dificuldade na classificação 
no reconhecimento de expressões, tendendo a reconhecer expressões neutras como negativas. Outros fatores como a TPM, que envolve
mudança de humor e comportamento, também são relatadas. Na menstruação, é comum que sintam sintomas físicos, como cólicas, demandem 
atividades físicas com menos esforço e prejudiquem a concentração. 

Tendo o conhecimento dessas influências, é proposto, então, um sistema de sugestão, utilizando os algoritmos de sistema de recomendação, 
que indicaria tarefas que seriam mais facilmente ou dificilmente realizadas dependendo do perfil e da fase do ciclo menstrual da pessoa. A fase 
do ciclo será determinada utilizando o método do calendário. Esse sistema poderá ajudar no auto-conhecimento de quem o utiliza, melhorando a inteligência emocial e fazendo com que 
as pessoas possam utilizar dos benefícios e lidar com os malefícios de cada fase de forma saudável e consciente.

No escopo deste trabalho, foi possível montar um processo de contagem de ciclo para determinar em que fase a pessoa se encontra. Foi utilizado 
o estudo de caso para realizar a coleta de dados a análise de resultados, sobre o perfil das mulheres e para determinar quais tarefas iriam demandar 
menos ou mais energia para serem executadas. O objetivo específico de desenvolver o aplicativo foi parcialmente cuprido e terá continuação no escopo do TCC2.

Ao longo do TCC1, foi possível vivenciar algumas atividades relevantes, tais como:

\begin{itemize}
    \item pesquisa e escrita científica;
    \item fechamento de escopo do trabalho;
    \item embasamento teórico e tecnológico;
    \item detalhamento metodológico;
    \item apresentação da proposta, das principais contribuições pretendidas com a realização da mesma e da prova de conceito, e
    \item acompanhamento contínuo das realizações, com reuniões periódicas entre orientado e orientador.
\end{itemize}

As atividades que envolverão o desenvolvimento do aplicativo bem como a análise dos resultados finais serão executadas no escopo do TCC2.