\chapter[Suporte Tecnológico]{Suporte Tecnológico}
\label{ch:suporte}

Neste capítulo, serão apresentadas, brevemente, as ferramentas e tecnologias 
utilizadas no desenvolvimento do aplicativo, no gerenciamento 
do projeto, no gerenciamento de configuração, no gerenciamento da 
escrita e na condução da pesquisa.


\section{Desenvolvimento do Aplicativo}

Esta seção contemplará as tecnologias que foram utilizadas na construção do 
código do aplicativo.

\subsection{React Native 0.68}

O React Native \cite{reactNative}, criado pelo Facebook, combina as melhores partes de 
desenvolvimenta nativo com 
o \emph{framework} React, uma biblioteca JavaScript para construir interface de usuário(UI). 
É usada para desenvolver aplicativos para iOS e Android de forma nativa, ou seja, utilizando 
componentes próprios da plataforma iOS ou Android para construir a aplicação. O React Native 
possibilita que uma única base de código possa ser utilizada para construir aplicações 
multi-plataforma.

Ele é composto por um SDK(\emph{Software Development Kit}) e um
\emph{framework}. O SDK é uma coleção de ferramentas que ajudam
o desenvolvedor a desenvolver a aplicação e executá-la em
plataformas específicas. Essas ferramentas incluem bibliotecas,
documentação, exemplos de códigos, processos, guias,
compiladores, entre outras coisas. Já o \emph{framework} é
uma coleção de elementos da UI que são reutilizáveis e podem
ser personalizados para as necessidades específicas da
aplicação \cite{reactNative}.

\subsection{TypeScript 4.4.4}

O TypeScript \cite{typescript} é uma linguagem de programação de código aberto, desenvolvida 
pela Microsoft, sendo construída em JavaScript, e fortemente tipada. Uma tipagem estática pode 
ser usada para descrever a forma de um objeto, o que possibilita uma melhor documentação do 
código, e permite que o TypeScript faça validações para garantir que o código esteja funcionando 
de forma adequada.

\subsection{Firebase}

O Firebase \cite{firebase2011} é uma plataforma desenvolvida pelo Google para a criação de aplicativos 
\emph{web} e móveis. Era originalmente uma empresa independente, fundada em 2011. 
Em 2014, o Google adquiriu a plataforma e, agora, é sua ferramenta principal 
para o desenvolvimento de aplicativos. O Firebase contêm funcionalidades como: 
análises; bancos de dados; mensagens e relatórios de erros, garantindo mais agilidade no desenvolvimento de aplicativos.

\section{Engenharia de Software}

Esta seção apresentará os suportes tecnológicos associados mais especificamente 
à Engenharia de Software. A mesma foi dividida em três subseções, sendo elas: 
Gerenciamento de Projetos, Gerenciamento de Desenvolvimento e Gerenciamento de 
Configuração.

\subsection{Gerenciamento do Projeto}

Esta subseção descreve as metodologias e ferramentas adotadas para o gerenciamento 
do projeto, tanto na escrita quanto no desenvolvimento da aplicação.

\begin{itemize}

    \item \textbf{Trello: }uma aplicação \emph{web} baseada no sistema Kanbam, que auxilia no 
	gerenciamento de tarefas para times grandes ou pessoas individuais. 
	Originalmente criado pela Fog Creek Software, em 2011, e vendida à Atlassian, em 
	2017 \cite{trello2011}. O Trello foi utilizado para organizar as tarefas 
	relacionadas à escrita do TCC, e também para manter o registro dos artigos utilizados;
    \item \textbf{ZenHub: }é uma ferramenta semelhante ao Trello. É exclusiva para uso junto ao 
	Github, sendo uma ferramenta adequada, focada em: rastreamento; planejamento e 
	relatórios das \emph{features} de projetos no GitHub. O Zenhub é baseado nas 
	metodologias ágeis, como Scrum, sendo utilizado, portanto, em projetos ágeis. Com ele, 
	é possível planejar roteiros; usar quadros de tarefas, e gerar relatórios 
	automatizados diretamente do repositório do GitHub \cite{zenhub2020}. 
	Nesse projeto, foi utilizado para organizar as tarefas relacionadas ao 
	desenvolvimento do aplicativo, e
    \item \textbf{Slack: }uma plataforma de comunicação que permite a criação de times e a 
	organização de canais de conversas por tópicos, grupos privados ou mensagens 
	diretas. Ele também possui integração com  Google Drive, Trello, Dropbox, Box, 
	Heroku, IBM Bluemix, Crashlytics, GitHub, entre outros \cite{slack2013}. Nesse 
	projeto, foi utilizado para comunicação entre aluno e orientador.

\end{itemize}


\subsection{Gerenciamento de Desenvolvimento}

Esta subseção descreve as ferramentas adotadas para a construção do código da aplicação.

\begin{itemize}

    \item \textbf{Visual Studio Code 1.63.2: }um editor de texto ou código fonte feito pela Microsoft, 
	bastante utilizado no desevolvimento de software \cite{vscode2015}. Tem suporte 
	para várias linguagens, e possui ferramentas importantes que auxiliam no 
	desenvolvimento de software, como por exemplo: \emph{debugger}, auto completar em 
	códigos, com sugestões inteligentes, e uma infinidade de extensões que podem ser instaladas 
	para auxiliar ainda mais no desenvolvimento. Foi usado como editor de código fonte 
	principal no trabalho;
    \item \textbf{Android Studio 2021.1.1: }ambiente de desenvolvimento para 
	aplicativos Androids oficial do Google. Desenvolvido pelo Google e pela 
	JetBrains, o ambiente possui ferramentas como editor de texto, editor de \emph{layout}, 
	analizador apk, rápidos emuladores, \emph{debugger}, entre outras ferramentas 
	que auxiliam no desenvolvimento de aplicativos \cite{android2020}. Utilizado no 
	trabalho como emulador de dispositivos Android;
    \item \textbf{MacOS: }sistema operacional desenvolvido e distribuído pela empresa Apple
	destinado exclusivamente para computadores Mac \cite{apple}. 
	Foi usado como principal sistema operacional nesse trabalho, e
	\item \textbf{Figma: }software para desenvolvimento de protótipos e 
	criação de projetos que não precisa de instalação, uma vez que está disponível 
	em nuvem, através de uma página web \cite{figma2020}. Possui \emph{features} para projetos, prototipação, 
	projetos de sistemas, colaboração e \emph{download}. Os protótipos contam com 
	interações, transições avançadas com animações inteligentes, GIFs animados, 
	entre outros. Tais recursos possibilitam uma experiência muito próxima ao mundo 
	real. Ele permite o compartilhamento rápido do protótipo a outras pessoas, e 
	facilita a contribuição. Nesse trabalho, foi utilizado como ferramenta para a 
	prototipação do aplicativo. 

\end{itemize}

\subsection{Gerenciamento de Configuração}

Esta subseção descreve as ferramentas adotadas para o versionamento do código da aplicação e da monografia.

\begin{itemize}

    \item \textbf{Git 2.30.1: }ferramenta gratuita e de código livre para controle 
	de versão, utilizada para lidar desde pequenos a grandes projetos com rapidez e 
	eficiência. Ele possibilita a criação de diversas branches que se ramificam e 
	podem ser editadas, combinadas e removidas sem sofrer perdas de dados. O controle de 
	versão também possibilita retornar a um momento específico do projeto e observar as 
	mudanças entre as versões \cite{git2020}. O Git 2.30.1 foi utilizado para versionamento de código fonte 
	do aplicativo desenvolvido nesse trabalho 
	e para a parte escrita do TCC, e
	\item \textbf{GitHub: }plataforma \emph{online} de desenvolvimento de 
	software que permite hospedar códigos e utilizar o controle de versão do Git. 
	Possibilita a revisão de códigos, gerenciamento de projetos, integração contínua, 
	hospedagem, integração de ferramentas, gerenciamento de equipe, documentação, 
	hospedagem de código, entre outras funcionalidades. Os repositórios, locais 
	onde os projetos são guardados, podem ser fechados ou abertos e possibilitam o 
	trabalho em equipe \cite{github2020}. Foi utilizado como plataforma que hospeda o código, 
	a documentação e a monografia do projeto.
    
\end{itemize}

\section{Escrita e Condução da Pesquisa}

\subsection{LaTex}

LaTex \cite{latex2020} é um sistema para preparação de documentos. Inclui 
recursos destinados à produção de documentação técnica e científica. O LaTex é 
utilizado como padrão para comunicação e publicação de documentos científicos, estando 
disponível como software livre. O LaTex possibilita a composição de artigos de 
periódicos, relatórios técnicos, livros e apresentações de \emph{slides}, faz 
o controle automático de seções, referências, tabelas, figuras, notas de rodapé, 
índices, entre outros, e possibilita a tipografia de fórmulas matemáticas 
complexas. Neste trabalho, foi utilizado localmente, junto ao editor de texto 
padrão do sistema operacional. O \emph{template} utilizado é o disponibilizado pela 
Faculdade do Gama no repositório do GitHub \footnote{Repositório do Github com o Template: https://github.com/fga-unb/template-latex-tcc}.
A versão utilizada foi a TeX Live 2017/Debian.


\section{Considerações Finais do Capítulo}


Neste capítulo, há uma breve descrição de todas as ferramentas  
utilizadas no desenvolvimento do projeto. Um resumo é apresentado na Tabela \ref{tab06}.

No desenvolvimento do aplicativo, foi utilizado o MacOS como sistema 
operacional; o IntelliJ IDEA como editor de texto; o Android Studio como 
o emulador de dispositivos Android, e o XCode como emulador de dispositivos IOS. 

No aplicativo, foi utilizado o React Native 
como \emph{framework} para criação de aplicativos nativos; o TypeScript como linguagem 
de programação, utilizada com o React Native; o Firebase Authetication para controle de acesso; 
o Firestore, também do Firebase, como banco de dados não relacional; o Cloud Functions, também do Firebase,
como serviço \emph{backend}, e o Javascript 
como linguagem de programação \emph{backend}.

As partes envolvendo o código, documentação e escrita do trabalho estão disponíveis 
em um repositório do Github que utiliza o Git para controle de versão.

O gerenciamento das tarefas referentes ao desenvolvimento do aplicativo foi realizado através do Zenhub, 
aplicativo disponível como extensão para o Github. O 
Trello foi utilizado para o gerenciamento das tarefas do referentes à escrita do TCC e 
rastreamento dos artigos. 

A escrita do TCC foi realizada utilizando o LaTex, com base em um \emph{template} 
disponibilizado pela Faculdade do Gama em um repositório do GitHub. 


\begin{table}[ht]
	\centering
    \caption{Suporte Tecnológico}
	\label{tab06}
	\begin{tabular}{ccc}
	\toprule
	Tecnologia & Utilização  & Versão   \\  
	\midrule
	\begin{minipage} [t] {0.3\textwidth} \centering  React Native \end{minipage}                     	& \begin{minipage} [t] {0.4\textwidth} \centering \emph{Framework} para criação de aplicativos nativos. \end{minipage}	 & \begin{minipage} [t] {0.2\textwidth} \centering  0.68 \end{minipage}   \\  
\midrule	\begin{minipage} [t] {0.3\textwidth} \centering  TypeScript \end{minipage}  & \begin{minipage} [t] {0.4\textwidth} \centering  Linguagem de programação.     \end{minipage}  &  \begin{minipage} [t] {0.2\textwidth} \centering  4.4.4 \end{minipage} \\  
\midrule	\begin{minipage} [t] {0.3\textwidth} \centering  Firebase  \end{minipage}                        & \begin{minipage} [t] {0.4\textwidth} \centering  Banco de dados, autenticação e funções.  \end{minipage}                   & \begin{minipage} [t] {0.2\textwidth} \centering  - \end{minipage}  \\  
\midrule	\begin{minipage} [t] {0.3\textwidth} \centering  Zenhub  \end{minipage} & \begin{minipage} [t] {0.4\textwidth} \centering  Gerenciamento de tarefas relacionadas ao desenvolvimento da aplicação.  \end{minipage}  & \begin{minipage} [t] {0.2\textwidth} \centering  - \end{minipage}  \\  
\midrule	\begin{minipage} [t] {0.3\textwidth} \centering Trello  \end{minipage}  	& \begin{minipage} [t] {0.4\textwidth} \centering Gerenciamento de tarefas relativas ao desenvolvimento dos TCCs e registro de artigos. \end{minipage}	 & \begin{minipage} [t] {0.2\textwidth} \centering  - \end{minipage}  \\  
\midrule	\begin{minipage} [t] {0.3\textwidth} \centering  Slack \end{minipage}    & \begin{minipage} [t] {0.4\textwidth} \centering  Comunicação com os orientadores. \end{minipage}        & \begin{minipage} [t] {0.2\textwidth} \centering  - \end{minipage}  \\  
\midrule	\begin{minipage} [t] {0.3\textwidth} \centering  Visual Studio Code \end{minipage}  	& \begin{minipage} [t] {0.4\textwidth} \centering Ambiente de desenvolvimento. \end{minipage}	 & \begin{minipage} [t] {0.2\textwidth} \centering  1.63.2 \end{minipage}  \\ 
\midrule	\begin{minipage} [t] {0.3\textwidth} \centering  Android Studio  \end{minipage} & \begin{minipage} [t] {0.4\textwidth} \centering  Geração de dispositivos virtuais android. \end{minipage}  &  \begin{minipage} [t] {0.2\textwidth} \centering  2021.1.1 \end{minipage}  \\ 
\midrule	\begin{minipage} [t] {0.3\textwidth} \centering  MacOS \end{minipage}                & \begin{minipage} [t] {0.4\textwidth} \centering  Sistema operacional.      \end{minipage}                   & \begin{minipage} [t] {0.2\textwidth} \centering  - \end{minipage}                               \\ 
\midrule	\begin{minipage} [t] {0.3\textwidth} \centering  Figma  \end{minipage} & \begin{minipage} [t] {0.4\textwidth} \centering  Prototipação. \end{minipage}  &  \begin{minipage} [t] {0.2\textwidth} \centering  - \end{minipage}  \\ 
\midrule	\begin{minipage} [t] {0.3\textwidth} \centering Git \end{minipage} 	& \begin{minipage} [t] {0.4\textwidth} \centering Controle de versão de código e escrita dos TCCs. \end{minipage}	 & \begin{minipage} [t] {0.2\textwidth} \centering  2.30.1 \end{minipage}   \\ 
\midrule	\begin{minipage} [t] {0.3\textwidth} \centering  \end{minipage} & \begin{minipage} [t] {0.4\textwidth} \centering  Repositório da aplicação e TCCs.   \end{minipage}  &  \begin{minipage} [t] {0.2\textwidth} \centering - \end{minipage}  \\ 
\midrule	\begin{minipage} [t] {0.3\textwidth} \centering  LaTex \end{minipage}   & \begin{minipage} [t] {0.4\textwidth} \centering  Edição de texto.  \end{minipage}  & \begin{minipage} [t] {0.2\textwidth} \centering  - \end{minipage}   \\ 
	\bottomrule
\end{tabular}
\end{table}
