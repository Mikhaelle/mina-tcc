\chapter[Suporte Tecnológico]{Suporte Tecnológico}
\label{ch:suporte}

Neste capítulo, há uma breve descrição de todas as ferramentas que foram 
utilizadas no desenvolvimento do projeto. 

No desenvolvimento do aplicativo, foi utilizado o MacOS como sistema 
operacional; o IntelliJ IDEA como editor de texto; o Android Studio como 
o emulador de dispositivos Android e o XCode como emulados de dispositivos IOS. 
Na aplicação, foi utilizado o React Native 
como \emph{framework} para criação de aplicativos nativos; o TypeScript como linguagem 
de programação, utilizada com o React Native; o Firebase Authetication para controle de acesso; 
o Firestore, também do Firebase, como banco de dados não relacional; o Cloud Functions, também do Firebase,
como serviço \emph{backend} e o Javascript 
como linguagem de programação \emph{backend}.

As partes envolvendo o código, documentação e escrita do trabalho estão disponíveis 
no Github que utiliza o Git para controle de versão.

O gerenciamento das tarefas referentes ao desenvolvimento do aplicativo foi realizado através do Zenhub, 
aplicativo disponível como extensão para o Github, 
e o Trello foi utilizado para o gerenciamento das tarefas do referentes a escrita do TCC1 e TCC2 e 
rastreamento dos artigos. 

A escrita do TCC foi realizada utilizando o LaTex com base em um \emph{template} 
disponibilizado pela Faculdade do Gama em um repositório do GitHub. 

\begin{table}[]
\begin{tabular}{|c|c|c|c|c}
\cline{1-4}
\cellcolor[HTML]{C0C0C0}Tecnologia                                                                  & \cellcolor[HTML]{C0C0C0}O que é                                                                                                   & \cellcolor[HTML]{C0C0C0}Utilização                                    & \cellcolor[HTML]{C0C0C0}Versão &  \\ \cline{1-4}
\begin{minipage} [t] {0.3\textwidth} \centering  React Native \cite{reactNative} \end{minipage}                     & \begin{minipage} [t] {0.3\textwidth} \centering  Bibliotéca Javascript criada pelo Facebook.  \end{minipage} 	& \begin{minipage} [t] {0.2\textwidth} \centering \emph{Framework} para criação de aplicativos nativos \end{minipage}	 & \begin{minipage} [t] {0.1\textwidth} \centering  0.68 \end{minipage}  &  \\ \cline{1-4}
\cellcolor[HTML]{EFEFEF}\begin{minipage} [t] {0.3\textwidth} \centering  TypeScript \cite{typescript}\end{minipage} & \cellcolor[HTML]{EFEFEF}\begin{minipage} [t] {0.3\textwidth} \centering  Linguagem de programação de código aberto desenvolvida pela Microsoft.   \end{minipage}                                              & \cellcolor[HTML]{EFEFEF}\begin{minipage} [t] {0.2\textwidth} \centering  Linguagem de programação     \end{minipage}  & \cellcolor[HTML]{EFEFEF} \begin{minipage} [t] {0.1\textwidth} \centering  4.4.4 \end{minipage} &  \\ \cline{1-4}
\begin{minipage} [t] {0.3\textwidth} \centering  Firebase \cite{firebase2011} \end{minipage}                   & \begin{minipage} [t] {0.3\textwidth} \centering  Plataforma desenvolvida pelo Google para a criação de aplicativos \emph{web} e móveis\end{minipage}                & \begin{minipage} [t] {0.2\textwidth} \centering  Banco de dados, autenticação e funções  \end{minipage}                   & \begin{minipage} [t] {0.1\textwidth} \centering  - \end{minipage} &  \\ \cline{1-4}
\begin{minipage} [t] {0.3\textwidth} \centering Trello \cite{trello2011} \end{minipage}                     & \begin{minipage} [t] {0.3\textwidth} \centering  Aplicação \emph{web} baseada no sistema Kanbam, que auxilia no gerenciamento de tarefas para times grandes ou pessoas individuais.   \end{minipage} 	& \begin{minipage} [t] {0.2\textwidth} \centering Gerenciamento de tarefas relativas ao desenvolvimento dos TCCs e registro de artigos. \end{minipage}	 & \begin{minipage} [t] {0.1\textwidth} \centering  - \end{minipage} &  \\ \cline{1-4}
\end{tabular}
\end{table}


\begin{table}[]
	\begin{tabular}{|c|c|c|c|c}
	\cline{1-4}
	\cellcolor[HTML]{EFEFEF}\begin{minipage} [t] {0.3\textwidth} \centering  Zenhub \cite{zenhub2020} \end{minipage} & \cellcolor[HTML]{EFEFEF}\begin{minipage} [t] {0.3\textwidth} \centering   Ferramenta adequada, focada em: rastreamento; planejamento e relatórios das \emph{features} de projetos no GitHub.   \end{minipage}             & \cellcolor[HTML]{EFEFEF}\begin{minipage} [t] {0.2\textwidth} \centering  Gerenciamento de tarefas relacionadas ao desenvolvimento da aplicação.  \end{minipage}  & \begin{minipage} [t] {0.1\textwidth} \centering  - \end{minipage} &  \\ \cline{1-4}
	\begin{minipage} [t] {0.3\textwidth} \centering  Slack \cite{slack2013} \end{minipage}                   & \begin{minipage} [t] {0.3\textwidth} \centering Plataforma de comunicação que permite a criação de times e a organização de canais de conversas por tópicos, grupos privados ou mensagens diretas.  \end{minipage}                & \begin{minipage} [t] {0.2\textwidth} \centering  Comunicação com os orientadores \end{minipage}        & \begin{minipage} [t] {0.1\textwidth} \centering  - \end{minipage} &  \\ \cline{1-4}
	\begin{minipage} [t] {0.3\textwidth} \centering  IntelliJ IDEA \cite{intelij} \end{minipage}                     & \begin{minipage} [t] {0.3\textwidth} \centering  Ambiente de desenvolvimento integrado desenvolvido pela JetBrains \end{minipage} 	& \begin{minipage} [t] {0.2\textwidth} \centering Ambiente de desenvolvimento \end{minipage}	 & \begin{minipage} [t] {0.1\textwidth} \centering  2021.3.1 \end{minipage}  &  \\ \cline{1-4}
	\cellcolor[HTML]{EFEFEF}\begin{minipage} [t] {0.3\textwidth} \centering  Android Studio \cite{android2020} \end{minipage} & \cellcolor[HTML]{EFEFEF}\begin{minipage} [t] {0.3\textwidth} \centering Ambiente de desenvolvimento para aplicativos androids oficial do Google. \end{minipage}                                              & \cellcolor[HTML]{EFEFEF}\begin{minipage} [t] {0.2\textwidth} \centering  Geração de dispositivos virtuais android \end{minipage}  & \cellcolor[HTML]{EFEFEF} \begin{minipage} [t] {0.1\textwidth} \centering  2021.1.1 \end{minipage} &  \\ \cline{1-4}
	\begin{minipage} [t] {0.3\textwidth} \centering  MacOS \end{minipage}                   & \begin{minipage} [t] {0.3\textwidth} \centering  Plataforma desenvolvida pelo Google para a criação de aplicativos \emph{web} e móveis.\end{minipage}                & \begin{minipage} [t] {0.2\textwidth} \centering  Banco de dados e autenticação .      \end{minipage}                   & \begin{minipage} [t] {0.1\textwidth} \centering  - \end{minipage}                              &  \\ \cline{1-4}
	\begin{minipage} [t] {0.3\textwidth} \centering  Xcode \end{minipage}                   & \begin{minipage} [t] {0.3\textwidth} \centering  Plataforma desenvolvida pelo Google para a criação de aplicativos \emph{web} e móveis.\end{minipage}                & \begin{minipage} [t] {0.2\textwidth} \centering  Banco de dados e autenticação .      \end{minipage}                   & \begin{minipage} [t] {0.1\textwidth} \centering  - \end{minipage}                              &  \\ \cline{1-4}
	\begin{minipage} [t] {0.3\textwidth} \centering Git \cite{git2020} \end{minipage}                     & \begin{minipage} [t] {0.3\textwidth} \centering  Ferramenta gratuita e de código livre para controle de versão.  \end{minipage} 	& \begin{minipage} [t] {0.2\textwidth} \centering Controle de versão de código e escrita dos TCCs. \end{minipage}	 & \begin{minipage} [t] {0.1\textwidth} \centering  2.30.1 \end{minipage}  &  \\ \cline{1-4}
	\cellcolor[HTML]{EFEFEF}\begin{minipage} [t] {0.3\textwidth} \centering  GitHub \cite{github2020} \end{minipage} & \cellcolor[HTML]{EFEFEF}\begin{minipage} [t] {0.3\textwidth} \centering Plataforma \emph{online} de desenvolvimento de software que permite hospedar códigos e utilizar o controle de versão do Git. \end{minipage}                                              & \cellcolor[HTML]{EFEFEF}\begin{minipage} [t] {0.2\textwidth} \centering  Repositório da aplicação e TCCs.   \end{minipage}  & \cellcolor[HTML]{EFEFEF} \begin{minipage} [t] {0.1\textwidth} \centering  2020.3.1 \end{minipage} &  \\ \cline{1-4}
	\cellcolor[HTML]{EFEFEF}\begin{minipage} [t] {0.3\textwidth} \centering  LaTex \cite{latex2020} \end{minipage} & \cellcolor[HTML]{EFEFEF}\begin{minipage} [t] {0.3\textwidth} \centering O LaTex é utilizado como padão para comunicação e publicação de documentos científicos, estando disponível como software livre. \end{minipage}    & \cellcolor[HTML]{EFEFEF}\begin{minipage} [t] {0.2\textwidth} \centering  Utilizado localmente para edição de texto utilizando o template padrão disponibilizado pela Faculdade do Gama.  \end{minipage}  & \cellcolor[HTML]{EFEFEF} \begin{minipage} [t] {0.1\textwidth} \centering  TeX Live 2017/Debian \end{minipage} &  \\ \cline{1-4}
\end{tabular}
\end{table}

\section{Considerações Finais do Capítulo}

Neste capítulo, foram aprensentadas brevemente as ferramentas e tecnologias 
utilizadas no desenvolvimento do aplicativo, no gerenciamento 
do projeto, no gerenciamento de configuração, no gerenciamento da 
escrita e na condução da pesquisa.
