\begin{agradecimentos}

    Agradeço à toda minha família, a qual eu amo muito. Em especial, aos meus pais,
    Maria do Carmo e Edir, minha irmã Emanuelle, minha Tia Nilza e minha madrinha
    Delma, pelo amor e apoio incondicional durante a minha trajetória na Faculdade do Gama.
    
    Aos meus amigos, pelo apoio e troca de experiência. Em especial, à Martha, ao Marcos
    Vinícius e à Fabiana, por sempre me acompanharem nessa jornada. 
    
    Ao meu namorado, Rodrigo, pelo apoio e amor incondicional, por ter me dado
    suporte durante toda dificuldade e ter comemorado cada etapa avançada ao meu lado.
    
    Às mulheres que se disponibilizaram a participar do caso de estudo, à Letícia e às trabalhadoras do santuário Anandamayi pela inspiração que deu origem ao tema.
    
    À Universidade de Brasília por ter contribuído na minha formação profissional e
    como pessoa, mostrando-me um mundo novo e as ferramentas necessárias para a constru-
    ção de uma sociedade mais justa e igualitária. Em particular, à Faculdade do Gama.
    
    Aos meus orientadores, Milene Serrano e Maurício Serrano, pela inspiração, paciência e compromentimento 
    em guiar-me durante esse processo, sempre oferecendo suporte técnico e psicológico que fizeram 
    dessa jornada uma trajetória muito gratificante. 
    
    Agradeço a todos os cientistas que possibilitaram a realização desse trabalho e espero que 
    o meu trabalho também seja importante para que outras pessoas possam dar continuidade nesse 
    estudo. 
    
    Agradeço, por fim, a mim mesma pela persistência e coragem; por me dedicar por
    incontáveis horas a este sonho; por acreditar que é possível construir algo valioso com muita determinação; 
    por conseguir, mesmo diante de tantas
    dificuldades, finalizar esta etapa da minha vida.
    
    \end{agradecimentos}
    