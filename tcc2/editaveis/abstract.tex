\begin{resumo}[Abstract]
  \begin{otherlanguage*}{english}
   During the menstrual cycle women go through various hormonal changes that can lead to
 changes in mood, behavior and physical symptoms. These changes can influence 
 the willingness to perform certain daily tasks. To help with self-knowledge, improve emotional 
 intelligence and enable them to use the benefits and deal with
 harm in a healthy and conscious way, this work proposes the realization of an 
 application with a task suggestion system based on
 profile and phase of the menstrual cycle. This system makes use of the inversion counting algorithm
 in a collaborative recommender system, which will suggest tasks
 that can be performed with more ease or difficulty. With bibliographic reference
 it was possible to use the calendar method to determine the phase of the cycle and with the
 case study, it was possible to survey the profiles and daily tasks to be used in the
 recommendation system.
    \vspace{\onelineskip}
  
    \noindent 
    \textbf{Key-words}: 
    menstrual cycle. recommendation system. application. inversion count.
  \end{otherlanguage*}
 \end{resumo}
 