\begin{resumo}[Abstract]
  \begin{otherlanguage*}{english}
    During the menstrual cycle, women go through several hormonal changes that
    can lead to
    changes in mood, behavior and physical symptoms. These changes can influence the
    willingness to perform certain everyday tasks. This work implements an application 
    with a system of
    task suggestion based on
    profile and phase of the menstrual cycle, 
    for help in self-knowledge, to improve emotional intelligence
    of these people, and to enable them to enjoy the benefits and deal with the
    harm
    in a healthy and conscious way. This system employs of the inversion counting algorithm
    in a collaborative recommender system, trying to suggest tasks
    that can be performed with more ease or difficulty. 
    It was possible to use the calendar method to determine the phase of the cycle through 
    the bibliographic reference. On the other 
    hand, the profiles and daily tasks used by the recommendation system 
    was developed considering the results of a survey.
    \vspace{\onelineskip}
  
    \noindent 
    \textbf{Key-words}: 
    menstrual cycle. recommendation system. application. inversion count.
  \end{otherlanguage*}
 \end{resumo}
 