\chapter[Conclusão]{Conclusão}
\label{ch:conclusao}

Neste capítulo, serão apresentadas as considerações finais sobre esse Trabalho de
Conclusão de Curso. Serão retomados os objetivos gerais e específicos estabelecidos no
Capítulo \ref{ch:intro}, evidenciando se os mesmos foram atingidos ou não, com justificativas. Por fim,
têm-se a apresentação dos pontos de melhorias e possíveis trabalhos futuros a serem aplicados.

\section{Objetivos Alcançados}

O objetivo geral de desenvolver um aplicativo informativo e de recomendações de tarefas 
com base no perfil e 
nas fases do ciclo menstrual, no intuito de apoiar as mulheres na identificação de mudanças, 
foi concluído com a geração 
do aplicativo Mina.

Para atingir esse objetivo geral, alguns objetivos específicos foram atendidos ao longo da trajetória de 
realização do trabalho. 

A definição de um processo de coleta de dados sobre o perfil das mulheres de acordo com seus ciclos 
menstruais foi estabelecido através de um estudo de caso de um grupo formado por 22 mulheres em idade fértil. 
Ao final, foi possível coletar o \emph{feedback} de sete dessas usuárias que estiveram presentes 
desde o início da realização desse trabalho. Esse número está de acordo com o recomendado para um 
estudo de caso citado no referencial teórico, que é de 4 a 10 pessoas. A definição de um processo de 
análise de resultados foi estabelecido também pelo estudo de caso, gerando 
relatórios e análises sobre os questionários aplicados.

O processo de contagem do ciclo para determinar que fase a do ciclo a mulher se encontra foi estabelecido 
através do método do calendário e estudos sobres as fases, descritos no Capítulo \ref{ch:referencial}. Adicionalmente, foram 
aplicados à lógica do código que compõe o aplicativo Mina, mais especificamente na parte de geração do calendário.

Tendo os processos anteriores definidos, foi possível determinar qual fase do ciclo a mulher 
se encontra, e utilizar esse dado para fazer recomendações de tarefas que podem 
ser realizadas de forma mais fácil, neutra ou mais difícil, baseando-se na fase do ciclo. A 
recomendação é gerada 
através da inserção das usuárias em um grupo mais próximo, dado pelo algoritmo de contagem de 
inversão, caracterizando esse tipo de sistema de recomendação como um sistema colaborativo.

Através do desenvolvimento de uma interface e de algoritmos, foi possível a criação do aplicativo Mina, 
sendo esse um aplicativo de recomendação de tarefas baseado nos processos definidos anteriormente.

\section{Pontos de Melhoria}

Com o teste de usabilidade realizado, foi possível identificar alguns pontos de melhoria na aplicação. 
Alguns pontos foram percebidos pela própria autora, e outros elencados pelas participantes do estudo de caso. Pontos esses 
que devem ser levados em consideração para o aperfeiçoamento futuro do aplicativo.

Como a Autora tem a intenção de tornar esse projeto um projeto \emph{Open Source} as melhorias serão 
relatadas como \emph{issues} no repositório do github. São elas:

\begin{itemize}

    \item Habilitar o uso do \emph{dark mode};
    \item Realizar testes com iOS;
    \item Adicionar uma legenda explicando o significado das cores das bolas no calendário;
    \item Adicionar uma legenda explicando o significado dos ícones de previsão;
    \item Adicionar um botão para sair do questionário ao tentar respondê-lo novamente na tela de perfil;
    \item Diminuir a demora para geração e atualização das recomendações;
    \item Adicionar um botão de "tente novamente" ou um \emph{push}, na tela de previsão, para recarregar as previsões;
    \item Adicionar um tutorial, explicando o objetivo do aplicativo, como navegar e o significado dos ícones;
    \item Implementar notificações para lembrar as usuárias do envio de avaliações;
    \item Implementar código de testes no \emph{backend};
    \item Aumentar a segurança dos dados, e
    \item Estar de acordo com a Lei Geral de Proteção de Dados Pessoais.

\end{itemize}

\section{Trabalhos Futuros}

Apesar de concluído esse trabalho, a Autora consegue vislumbrar outras oportunidades para trabalhos futuros.

No universo de ciclo menstrual, há a possibilidade de tratar sobre sintomas físicos e emocionais decorrentes 
da flutuação hormonal do ciclo menstrual, além de estabelecer métodos mais criteriosos para determinar 
algumas fases do ciclo não tratadas nesse trabalho, como a janela do período fértil. Outra possibilidade seria 
monitorar o tamanho dos ciclos e tratar de ciclos irregulares através de métodos estatísticos para 
tentar estabelecer de forma mais precisa as fases para esses casos.

Quanto ao sistema de recomendação, é possível implementar algoritmos mais sofisticados para melhorar 
a precisão das recomendações. Um exemplo seria, com mais dados de usuárias já ativas no sistema, 
utilizar o algoritmo \emph{clustering} para plotar as usuárias em um gráfico, e determinar 
novos grupos a partir de pontos mais densos encontrados. Isso também possibilitaria o cruzamento com outros 
metadados que podem ser coletados, como métodos anticoncepcionais, idade, entre outros.

Outro ponto seria a possibilidade de deixar que as usuárias selecionem as tarefas que elas realmente 
realizam no dia a dia, e adicionem novas tarefas não listadas. Outros algoritmos de recomendação teriam 
que ser desenvolvidos para suportar novos itens ou personalizações pessoais. Nessa mesma linha de pensamento, 
um algoritmo híbrido poderia ser implementado dando mais peso às pontuações pessoais do que as do grupo.

Os escopos de atuação desse trabalho, tanto sobre o ciclo menstrual, quanto sobre o sistema de recomendação, 
são enormes, e ainda há muitas possibilidades a serem exploradas no futuro.
