\chapter*[Introdução]{Introdução}
\addcontentsline{toc}{chapter}{Introdução}

Neste capítulo, serão descritos a contextualização, apresentando brevemente o tema; a justificativa, 
apesentando os porquês da elaboração do trabalho; os objetivos geral e específicos e a organização dos 
capítulos dessa monografia.

\section{Contextualização}

O ciclo menstrual feminino começou a ser pesquisado, cientificamente, na década de 1930 \cite{frank1931}.
Desde essa iniciativa, com os avanços da ciência e da medicina, métodos cada vez mais sofisticados e 
acessíveis para análise hormonal possibilitaram também vários estudos nessa área que ainda intriga 
muitos cientistas da medicina, psicologia e a sociedade em geral.


Em 2005, estratégias e métodos foram estabelecidos para estudar o ciclo menstrual e obter a 
classificação correta das fases do ciclo \cite{becker2005}.
Essas estratégias e métodos utilizam-se de medidas hormonais, temperatura corporal basal(TCB) e 
avaliações baseadas em calendário. O ciclo menstrual idealizado tem 28 dias, mas pode variar entre 21 
e 35 dias \cite{lenton1984a}, começando a ser contabilizado a partir do primeiro dia da menstruação. 
O ciclo é dividido, principalmente, em 2 fases, sendo: a fase folicular e a fase lútea 
\cite{brondin2008}.


A fase folicular é contabilizada a partir do primeiro dia da menstruação até o dia de pico do hormônio 
luteinizante(LH). Essa fase é caracterizada pelo desenvolvimento folicular e, normalmente, tem o 
comprimento de 14 
dias, podendo ter variações dependendo da idade \cite{lenton1984a}.

Na fase folicular, um folículo é selecionado para se tornar um óvulo, e aumenta a produção do estradiol
que faz surgir o LH e a progesterona. O surgimento desses hormônios caracterizam clinicamente o ciclo 
ovulatório e o início da fase lútea \cite{fritz2010}. É nesse ciclo que ocorre o aumento da temperatura
basal e o óvulo pode ser fecundado. 


Na fase lútea, a progesterona prepara o endométrio para a chegada do óvulo no caso de concepção. 
Caso não haja fecundação, a progesterona decai progressivamente e causa novamente a menstruação, 
continuando assim o ciclo \cite{nikas2003}.


O estradiol e a progesterona são altamente lipofílicos, ou seja, se dissolvem em gordura, óleos e 
lipídios em geral, e facilmente atravessam a barreira sangue-cérebro. Estudos em animais e estudos 
post-mortem em mulheres na idade reprodutiva e na menopausa indicaram que esses hormônios estavam 
acumulados no cérebro \cite{bixo1997}. Os receptores desses hormônios estão presentes em áreas 
cerebrais associadas à reprodução, função cognitiva e processamento emocional, como o hipotálamo e o 
sistema límbico \cite{gruber2002, brinton2008}.


Tendo como teoria que os hormônios podem influenciar a vida das mulheres, muitos estudos têm sido 
realizados tentando determinar a influência da fase do ciclo menstrual na capacidade cognitiva, 
motora e emocional das mulheres. Em 2014, \citeonline{poroma2014} realizaram um levantamento da 
literatura existente que relacionam o ciclo reprodutivo feminino com as áreas de tarefas cognitivas, 
tais como: habilidades espacial, visual, verbal, controle cognitivo, e aspectos emocionais.


No estudo de \citeonline{poroma2014}, na parte de habilidade espacial visual, 
\citeonline{hausmann2000}, \citeonline{maki2002}, \citeonline{courvoisier2013}, \citeonline{becker1982} e \citeonline{phillips1992} relataram uma melhora 
nas habilidades no início da fase folicular. \citeonline{hampson2014} relatou melhora nas habilidades quando o 
estradiol estava baixo. 


Em tarefas verbais, o trabalho \cite{maki2002} relatou melhoras no meio da fase lútea. Outro 
reportou que há melhor realização de tarefas verbais em mulheres que utilizam anticoncepcional(AC) 
\cite{mordecai2008}. Por fim, dois trabalhos concluem que a realização das tarefas verbais é melhor 
ao final das fases folicular e lútea \cite{Rosenberg2002, solis2004}.


No aspecto emocional, vários estudos relacionaram as fases com a habilidade de reconhecer emoções 
faciais. Em \citeonline{gasbarri2008}, os resultados indicaram que esse reconhecimento foi mais 
preciso quando a progesterona estava alta. Entretanto, vários outros autores indicaram que o 
reconhecimento piora na fase lútea, principalmente em reconhecer emoções negativas \cite{gasbarri2008}.
Há ainda mais um estudo que correlacionou o aumento de hormônios do estresse com a fase lútea \cite{kirschbaum1999}.



Já nos aspectos emocionais e comportamentais, no estudo de \citeonline{rosa2016}, levantou-se que, 
durante o período da semana que antecede a menstruação e no período da menstruação, as mulheres 
entrevistadas relataram sentir uma maior alteração psicológica ou comportamental. Essas alterações 
compreendem variações de humor, irritabilidade, ansiedade, entre outros. Essas alterações podem ser 
classificadas como transtorno disfórico pré-menstrual (TDPM) \cite{ACOG2000}.

Esse estudo fundamenta-se no fato de que o ciclo menstrual influencia as emoções e os comportamentos 
das mulheres. Como contribuição principal, propõe-se um aplicativo de recomendação, o qual procurará 
indicar tarefas cotidianas que seriam mais fáceis ou mais difíceis de serem realizadas, dependendo do 
perfil e da fase do ciclo menstrual que a mulher se encontra, além de conferir informações sobre como 
a fase possivelmente influencia em suas habilidades. Espera-se que esse conhecimento beneficie, 
adicionalmente, a inteligência emocional e, consequentemente, o desempenho cognitivo individual das 
mulheres. 

A inteligência emocional é medida utilizando \textit{Multifactor Emotional Intelligence Scale} (MEIS), 
compreendendo três habilidades emocionais distintas: perceber, entender e regular emoções. Em linhas 
gerais, a inteligência 
emocional é \lq \lq A habilidade de monitorar os próprios sentimentos e emoções e de outros indivíduos, 
discriminar entre eles e usar essas informações para guiar o pensamento e ações \rq \rq \cite{salovey1990}. 
No estudo realizado por \citeonline{lam2002}, os autores relacionam o impacto da inteligência emocional 
com o desempenho individual e confirmam a hipótese de 
que a inteligência emocional influência no desempenho cognitivo individual. 

Diante dos exposto, para as mulheres, 
a possibilidade de adquirirem conhecimentos sobre como e porquê suas fases ciclicas influenciam suas vidas, 
prevendo padrões em seus comportamentos, poderia aumentar 
significativamente a inteligência emocional dessas mulheres. 


O uso de recomendação, mais precisamente o uso dos conceitos associados aos sistemas de recomendação, 
foi escolhido, pois esses sistemas tratam muitas informações, fornecendo recomendações personalizadas, 
e procurando equacionar fatores como: precisão, novidade, dispersão e estabilidade \cite{bobadilla2013}.


A problemática, portanto, compreende levantar os diversos tipos de perfis das mulheres de acordo com a 
fase do ciclo menstrual delas, descobrir se existe diferença entre mulheres que utilizam ou não métodos 
hormonais ou tem algum distúrbio hormonal, e como essas nuances influenciam as tarefas cotidianas delas. 
Além de conseguir inferir em qual fase do ciclo a mulher está apenas com o uso do método do calendário. 
Para isso, será utilizada a literatura existente sobre o assunto, pesquisas previamente publicadas e 
pesquisas da própria autora, em formato de questionários, sendo esses aplicados a mulheres em idade 
reprodutiva.


Dado a particularidade do assunto, bem como por lidar com aspectos pessoais, vale ressaltar que este 
estudo não tem como objetivo entrar na discussão de diferença entre sexo. Além disso, o estudo não 
conta com o acompanhamento dedicado de um profissional da saúde, uma vez que a ideia não é representar 
um tratamento médico ou algo nesse sentido. Trata-se apenas de um estudo, tendo como objetivo desenvolver 
um aplicativo informativo, totalmente baseado na literatura especializada, e que estimula as mulheres a 
se conhecerem melhor. 


\section{Questões de Pesquisa}

Com base no exposto na contextualização, o presente trabalho pretende colaborar com esses estudos, 
os quais relacionam as fases do ciclo menstrual com as diferentes habilidades inerentes nas tarefas 
cotidianas e profissionais das mulheres.

Nesse sentido, as seguintes questões de pesquisa nortearão o trabalho, sendo:

\begin{itemize}

        \item Se existe essa influência das fases do ciclo menstrual na vida das mulheres, como adquirir esse conhecimento?

        \item Por que seria importante adquirir esse conhecimento? 

        \item Como utilizar esse conhecimento de forma a auxiliar as mulheres nas suas tarefas cotidianas e profissionais?

\end{itemize}

\section{Justificativa}

Para uma mulher pode ser difícil monitorar, identificar padrões e antecipar mudanças físicas, 
emocionais e comportamentais que existem no decorrer do ciclo, e como isso influencia a vida delas.

Atualmente, existem muitos aplicativos no mercado que estão voltados para a questão reprodutiva, 
e alguns até possuem \textit{features} para adicionar cotidianamente os sintomas sentidos, humor, 
temperatura corporal, intensidade do fluxo, tipo de muco, medicamentos tomados, relações sexuais, 
entre outras notas. Entretanto, existe uma carência de recursos que auxiliem de forma informativa bem 
como com recomendações de tarefas baseadas no perfil e na fase do ciclo menstrual.

Apesar dos avanços, relativamente poucas descobertas que correlacionam a influência hormonal com a 
emoção, o comportamento e a cognição são de fato conclusivas \cite{poroma2014}. Portanto, esse estudo 
também trará um levantamento sobre as influências do ciclo menstrual relatadas por um grupo de mulheres 
em idade reprodutiva.


\section{Objetivos}

Seguem os objetivos Geral e Específicos atrelados a esse trabalho.

\subsection{Objetivo Geral}


Esse estudo propõe um aplicativo informativo e de recomendações de tarefas com base no perfil e no ciclo 
menstrual, no intuito de apoiar as mulheres na identificação de mudanças; no autoconhecimeno; na inteligência emocional, e na produtividade pessoal.


\subsection{Objetivos Específicos}

Para atingir o objetivo geral, alguns objetivos específicos foram estabelecidos:

\begin{itemize}

        \item Definir um processo de coleta de dados sobre o perfil das mulheres de acordo com seus ciclos menstruais;
        
        \item Definir um processo de análise de resultados obtidos a partir dos dados coletados;

        \item Definir um processo de contagem de ciclo para determinar em que fase do ciclo menstrual a mulher se encontra;

        \item Definir um processo de recomendação de tarefas que podem ser mais facilmente ou dificilmente realizadas baseado na fase do ciclo, e

        \item Desenvolver um aplicativo de recomendação de tarefas baseado nos processos definidos anteriormente.

\end{itemize}

\section{Organização dos Capítulos}

A monografia está organizada nos seguintes capítulos:

\begin{itemize}


        \item Capítulo 2 - Referencial Teórico: descreve os conceitos que fundamentam o trabalho e o conhecimento necessário para que se compreenda a pesquisa realizada;
        
        \item Capítulo 3 - Suporte Tecnológico: aborda os principais suportes tecnológicos que viabilizarão o desenvolvimento da proposta;
        
        \item Capítulo 4 - Metodologia: acorda o plano metodológico que orienta o presente trabalho em termos conceituais, bem como de análise de resultados;

        \item Capítulo 5 - Proposta: apresenta a proposta deste trabalho em si, e

        \item Capítulo 6 - Resultados Obtidos: acorda os resultados alcançados até o momento, no escopo do TCC-01.
          
\end{itemize}
% Como a concentração de estradiol e progesterona variam muito de mulher para mulher, utilizá-lo como medida única pode ser insuficiente para determinar individualmente a fase do ciclo \cite{poroma2014}.
% em 1973 Barbara Sommer revisou toda a lieratura existente e concluiu, naquela época, que não existia evidencias que a fase do ciclo menstrual influenciava em mudanças na cognição e performance motora por causa de problemas na metodolodia. Co
%Há uma diminuição significativa do comprimento da fase folicular de acordo com a idade das mulheres. Normalmente mulheres de 18 a 24 anos tem a fase com o comprimento de 14 dias e mulheres de 40 a 44 anos tem de 10 dias }. 
%A fase folicular Quando um folículo é selecionado o FSH diminui gradativamente e progressivamente a produção de estradiol começa a aumentar. Quando o estradiol chega ao pico, na fase folícular, 12 a 24 horas depois o LH surge e a ovulação ocorre tipicamente de 10 a 12h depois do surgimento do LH \cite{fritz2010}. Clinicamente é possível determinar o ciclo ovulatório pelo surgimento do LH e a secreção de progesterona da fase lútea \cite{fritz2010}, também é possível determiná-lo pelo aumento da temperatura basalcorporal. Na ovulação, o folículo se transforma num corpus luteum que é capaz de sintetizar estradiol e progesterona e está pronto para ser fecundado.
