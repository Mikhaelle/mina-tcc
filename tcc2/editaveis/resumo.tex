\begin{resumo}

    Durante o ciclo menstrual, pessoas passam por várias mudanças hormonais que podem levar a 
    mudanças de humor, comportamento e sintomas físicos. Essas mudanças podem influenciar  
    positivamente ou negativamente a realização de certas tarefas cotidianas. Para 
    ajudar no autoconhecimento, melhorar a inteligência emocional 
    dessas pessoas e possibilitar que elas utilizem dos benefícios e lidem com os 
    malefícios 
    de forma saudável e consciente, esse trabalho propõe a 
    realização de um aplicativo com um sistema de sugestão de tarefas baseado em 
    perfil e fase do ciclo menstrual. Esse sistema utiliza algoritmos de sistema 
    de recomendação, procurando sugerir tarefas 
    que demandam mais ou menos energia para serem executadas. Com o levantamento bibliográfico, 
    foi possível utilizar o método do calendário para determinar a fase do ciclo. Juntamente com o 
    estudo de caso, foi possível levantar os perfis e tarefas cotidianas a serem utilizados no sistema de 
    recomendação.
    
     \vspace{\onelineskip}
        
     \noindent
     \textbf{Palavras-chave}: ciclo menstrual. sistema de recomendação. aplicativo. influência das fases.
    \end{resumo}
    