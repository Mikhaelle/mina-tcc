\begin{resumo}

    Durante o ciclo menstrual, mulheres passam por várias mudanças hormonais que podem levar a 
    mudanças de humor, comportamento e sintomas físicos. Essas mudanças podem influenciar na 
    disposiçao para realizar certas tarefas cotidianas. Para 
    ajudar no autoconhecimento, melhorar a inteligência emocional 
    dessas pessoas, e possibilitar que elas usufruam dos benefícios e lidem com os 
    malefícios 
    de forma saudável e consciente, esse trabalho implementa um aplicativo com um sistema de 
    sugestão de tarefas baseado em 
    perfil e fase do ciclo menstrual. Esse sistema faz uso do algoritmo de contagem de inversões
    em um sistema de recomendação colaborativo, procurando sugerir tarefas 
    que podem ser executadas com mais facilidade ou dificuldade. Com o levantamento bibliográfico, 
    foi possível utilizar o método do calendário para determinar a fase do ciclo. Juntamente com o 
    estudo de caso, foi possível levantar os perfis e tarefas os perfis e as tarefas cotidianas, 
    sendo esses insumos utilizados no sistema de recomendação.
    
     \vspace{\onelineskip}
        
     \noindent
     \textbf{Palavras-chave}: ciclo menstrual. sistema de recomendação. aplicativo. contagem de inversões.
    \end{resumo}
    