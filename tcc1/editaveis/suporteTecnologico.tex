\chapter[Suporte Tecnológico]{Suporte Tecnológico}
\label{ch:suporte}

\section{Considerações Iniciais}
Neste capítulo, serão aprensentadas brevemente as ferramentas e tecnologias 
utilizadas no desenvolvimento do aplicativo, no gerenciamento 
do projeto, no gerenciamento de configuração, no gerenciamento da 
escrita e na condução da pesquisa.

\section{Desenvolvimento da Aplicação}

Esta seção contemplará as tecnologias que foram e serão ainda utilizadas na construção do 
código da aplicação.

\subsection{Flutter 1.22.2}

O Flutter \cite{flutter2017} é um kit de ferramentas de interface de usuário(UI) 
grátis e \emph{open-souce}, criado pelo Google e lançado em 2017. 
Ele auxilia na criação de aplicativos nativos para dispositivos 
\emph{mobile}, \emph{web} e \emph{desktop} a partir de uma 
única base de código. Isso significa que é possível criar uma 
aplicação para diferentes sistemas operacionais(IOS e Android) 
utilizando um único código \cite{flutter2017}.

Ele é composto por um SDK(\emph{Software Development Kit}) e um
\emph{framework}. O SDK é uma coleção de ferramentas que ajudam
o desenvolvedor a desenvolver a aplicação e executá-la em
plataformas específicas. Essas ferramentas incluem bibliotecas,
documentação, exemplos de códigos, processos, guias,
compiladores, entre outras coisas. Já o \emph{framework} é
uma coleção de elementos da UI que são reutilizáveis e podem
ser personalizados para as necessidades específicas da
aplicação \cite{flutter2017}.

\subsubsection{Dart 2.10.2}

O Dart \cite{flutter2017} é a linguagem de programação utilizada no Flutter, sendo também 
criada pelo Google em 2011. É uma linguagem focada para 
desenvolvimento \emph{front-end} e do tipo orientada a objetos.

\subsubsection{Firebase}

O Firebase \cite{firebase2011} é uma plataforma desenvolvida pelo Google para a criação de aplicativos 
\emph{web} e móveis. Era originalmente uma empresa independente, fundada em 2011. 
Em 2014, o Google adquiriu a plataforma e, agora, é sua ferramenta principal 
para o desenvolvimento de aplicativos. O Firebase contêm funcionalidades como: 
análises; bancos de dados; mensagens e relatórios de erros, garantindo mais agilidade no desenvolvimento de aplicativos.

\section{Engenharia de Software}

Esta seção apresentará os suportes tecnológicos associados mais especificamente 
à Engenharia de Software. A mesma foi dividida em três subseções, sendo elas: 
Gerenciamento de Projetos, Gerenciamento de Desenvolvimento e Gerenciamento de 
Configuração.

\subsection{Gerenciamento do Projeto}

Esta subseção descreve as metodologias e ferramentas adotadas para o gerenciamento 
do projeto, tanto na escrita quanto no desenvolvimento da aplicação.

\subsubsection{Trello}

O Trello \cite{trello2011} é uma aplicação \emph{web} baseada no sistema Kanbam, que auxilia no 
gerenciamento de tarefas para times grandes ou pessoas individuais. 
Originalmente criado pela Fog Creek Software, em 2011, e vendida à Atlassian, em 
2017. O Trello foi utilizado para organizar as tarefas 
relacionadas à escrita do TCC, e também para manter o registro dos artigos utilizados.

\subsubsection{ZenHub}

O Zenhub \cite{zenhub2020} é uma ferramenta semelhante ao Trello. É exclusiva para uso junto ao 
Github, sendo uma ferramenta adequada, focada em: rastreamento; planejamento e 
relatórios das \emph{features} de projetos no GitHub. Ele é baseado nas 
metodologias ágeis, como Scrum, e é utilizado em projetos ágeis. Com ele, 
é possível planejar roteiros; usar quadros de tarefas e gerar relatórios 
automatizados diretamente do repositório do GitHub. 
Nesse projeto, será utilizado para organizar as tarefas relacionadas ao 
desenvolvimento do aplicativo proposto.


\subsubsection{Slack}

O Slack \cite{slack2013} é uma plataforma de comunicação que permite a criação de times e a 
organização de canais de conversas por tópicos, grupos privados ou mensagens 
diretas. Ele também possui integração com  Google Drive, Trello, Dropbox, Box, 
Heroku, IBM Bluemix, Crashlytics, GitHub, entre outros. Nesse 
projeto, será utilizado para comunicação entre aluno e orientador.

\subsection{Gerenciamento de Desenvolvimento}

Esta subseção descreve as ferramentas adotadas para a construção do código da aplicação.

\subsubsection{Visual Studio Code 1.50.1}

Visual Studio Code é um editor de texto ou código fonte feito pela Microsoft, 
bastante utilizado no desevolvimento de software \cite{vscode2015}. Tem suporte 
para várias linguagens, e possui ferramentas importantes que auxiliam no 
desenvolvimento de software, como por exemplo: \emph{debugger}, auto completar em 
códigos, com sugestões inteligentes, e uma infinidade de extensões que podem ser instaladas para auxiliar ainda mais no desenvolvimento. Será usado como editor de código fonte principal no trabalho. 

\subsubsection{Android Studio 3.6.3}

O Android Studio \cite{android2020} é o ambiente de desenvolvimento para 
aplicativos androids oficial do Google. Desenvolvido pelo Google e pela 
JetBrains, o ambiente possui ferramentas como editor de texto, editor de \emph{layout}, 
analizador apk, rápidos emuladores, \emph{debugger}, entre outras ferramentas 
que auxiliam no desenolvimento de aplicativos. Apenas o emulador de dispositivo 
desse software será utilizado no trabalho por conter tanto dispositivos Android 
quanto IOS.

\subsubsection{Linux Mint}

O Linux Mint \cite{lm2020} é uma distribuição Linux gratuita e baseada no Ubuntu. O sistema operacional possui aplicações de código aberto ou código livre e é mantido pelo Linux Mint Team e a comunidade. Será usado como principal sistema operacional nesse trabalho.

\subsubsection{Figma}

O Figma \cite{figma2020} é um software para desenvolvimento de protótipos e 
criação de projetos que não precisa de instalação, uma vez que está disponível 
em nuvem, através de uma página web. Possui \emph{features} para projetos, prototipação, 
projetos de sistemas, colaboração e \emph{download}. Os protótipos contam com 
interações, transições avançadas com animações inteligentes, GIFs animados, 
entre outros. Tais recursos possibilitam uma experiência muito próxima ao mundo 
real. Ele permite o compartilhamento rápido do protótipo a outras pessoas e 
facilita a contribuição. Nesse trabalho, será utilizado como ferramenta para a 
prototipação do aplicativo proposto. 

\subsection{Gerenciamento de Configuração}

Esta subseção descreve as ferramentas adotadas para o versionamento do código da aplicação e da monografia.


\subsubsection{Git 2.17.1}

O Git \cite{git2020} é uma ferramenta gratuita e de código livre para controle 
de versão, utilizada para lidar desde pequenos a grandes projetos com rapidez e 
eficiência. Ele possibilita a criação de diversas branches que se ramificam e 
podem ser editadas, combinadas e removidas sem sofrer perdas de dados. O controle de 
versão também possibilita retornar a um momento específico do projeto e observar as 
mudanças entre as versões. O Git 2.17.1 é utilizado para versionamento de código fonte 
do aplicativo a ser desenvolvido nesse trabalho 
e para a parte escrita do TCC.


\subsubsection{GitHub}

O GitHub \cite{github2020} é uma plataforma \emph{online} de desenvolvimento de 
software que permite hospedar códigos e utilizar o controle de versão do Git. 
Possibilita a revisão de códigos, gerenciamento de projetos, integração continua, 
hospedagem, integrações de ferramentas, gerenciamento de equipe, documentação, 
hospedagem de código, entre outras funcionalidades. Os repositórios, locais 
aonde os projetos são guardados, podem ser fechados ou abertos e possibilitam o 
trabalho em equipe. Será utilizado como plataforma que hospedará o código, 
documentação e monografia do projeto

\section{Escrita e Condução da Pesquisa}

\subsection{LaTex}

LaTex \cite{latex2020} é um sistema para preparação de documento. Inclui 
recursos destinados à produção de documentação técnica e científica. O LaTex é 
utilizado como padão para comunicação e publicação de documentos científicos, estando 
disponível como software livre. O LaTex possibilita a composição de artigos de 
periódicos, relatórios técnicos, livros e apresentações de \emph{slides}, faz 
o controle automático de seções, referências, tabelas, figuras, notas de rodapé, 
índices, entre outros e possíbilita a tipografia de fórmulas matemáticas 
complexas. Neste trabalho, é utilizado localmente, junto ao editor de texto 
padrão do sistema operacional. O \emph{template} utilizado é o disponibilizado pela 
Faculdade do Gama no repositório do GitHub \footnote{Repositório do Github com o Template: https://github.com/fga-unb/template-latex-tcc}.
A versão utilizada foi a TeX Live 2017/Debian.

\section{Considerações Finais do Capítulo}

Neste capítulo, há uma breve descrição de todas as ferramentas que serão 
utilizadas no desenvolvimento do projeto. 

No desenvolvimento do aplicativo, será utilizado o Linux Mint como sistema 
operacional; o Visual Studio Code como editor de texto, e o Android Studio como 
o emulador de dispositivos IOS e Android. Na aplicação, será utilizado o Flutter 
como \emph{framework} para criação de aplicativos nativos; o Dart como linguagem 
de programação do Flutter, e o Firebase como banco de dados e controle de acesso. 

As partes envolvendo o código, documentação e escrita do trabalho serão disponibilizadas 
no Github que utiliza o Git para controle de versão.

O gerenciamento do projeto do aplicativo tem sido executada através do Zenhub, aplicativo disponível no Github, e o Trello tem sido utilizado para o gerenciamento das tarefas do TCC. 

A escrita do TCC tem sido realizada utilizando o LaTex com base em um \emph{template} 
disponibilizado pela Faculdade do Gama em um repositório do GitHub. 

