\chapter[Suporte Tecnológico]{Suporte Tecnológico}

\section{Desenvolvimento da Aplicação}

\subsection{Flutter}

\subsection{Dart}

\subsection{back}

\section{Engenharia de Software}

\subsection{Gerenciamento do Projeto}

\subsubsection{Trello}

A ferramento Trello3 possibilita de modo fácil, gratuito e flexível o gerenciamento de projetos (TRELLO, 2017). O Trello auxilia no ganho de produtividade tanto por equipes grandes quanto de forma individual. Baseia-se no sistema de kanban4, bastante utilizado no desenvolvimento com Scrum, metodologia utilizada neste traballho (TRELLO,2017). O Trello foi utilizado para organizar as tarefas envolvidas durante a pesquisa, desenvolvimento e escrita deste trabalho.

\subsubsection{Slack}

\subsection{Gerenciamento de Desenvolvimento}

\subsubsection{Visual Studio Code}

Visual Studio Code (VSCODE, 2017) é um editor de texto ou código fonte da Microsoft(MICROSOFT, 2017) que possui ferramentas poderosas que auxiliam no desenvolvimento de software, como por exemplo conclusão e depuração de código IntelliSense(Sugestões inteligentes de auto preenchimento de algum parâmetro ou atributo no código). Além disso, o editor suporta diversas linguagens de programação, e possui um acervo grande de plugins, visando tornar seu trabalho mais eficiente

\subsubsection{Linux Mint}
(UBUNTU, 2017) é uma distribuição Linux totalmente gratuita e de código aberto. Este sistema operacional, patrocinado pela Canonical(CANONICAL, 2017), foi desenvolvido utilizando o kernel linux em seu núcleo.

\subsection{Gerenciamento de Configuração}

\subsubsection{Git}
Git é uma ferramenta para controle de versão distribuída sob a licença GNU General Public License version 2.0, uma licença open source. Traz benefícios como velocidade, garantia da integridade dos dados e suporte para fluxos de trabalho distribuídos e não-lineares (CHACON; STRAUB, 2014, pág. 31). Por estas razões, o Git versão 2.7.4 foi utilizado para versionamento do código fonte e para a parte escrita do TCC.

\subsubsection{GitHub}

O GitHub é uma plataforma de desenvolvimento de software. Nesta plataforma, é possível hospedar e analisar códigos, gerenciar projetos, e construir software colaborando com outros desenvolvedores. O GitHub apresenta funcionalidades que apoiam: revisão de código; gerenciamento de projetos; integrações de ferramentas; gerenciamento de equipe; codificação social; documentação, e hospedagem de código. Por se tratar de uma ferra-menta online, a versão do Github utilizada foi a disponível durante o desenvolvimento dotrabalho.

\section{Escrita e Condução da Pesquisa}

\subsection{LaTex}
LaTeX 8 é um sistema de composição de documentos de alta qualidade, que inclui recursos projetados para a produção de documentação técnica e científica. LaTeX é o padrão para a comunicação e publicação de documentos científicos, sendo disponibilizada como software livre. As principais funcionalidades da ferramenta utilizadas foram:
(i) Composição de artigos de periódicos, relatórios técnicos, livros e apresentações de slides;
(ii) Controle sobre documentos contendo seções, referências cruzadas, tabelas e figuras;
(iii) Tipografia de fórmulas matemáticas complexas; (iv) Geração automática de biblio-grafias e índices; e (v) Formatação multilíngue.

Neste trabalho, o LaTeX serviu de apoio à escrita, fazendo uso de todas as suas funcionalidades disponíveis nesse sentido. A ferramenta foi utilizada por intermédio doOverleaf.
\section{Resumo do Capítulo}

