\chapter[Metodologia]{Metodologia}

\section{Considerações Iniciais}

Neste capítulo, será apresentada a metodologia utilizada no desenolvivmento dessa 
monografia. Na seção 4.1 será detalhado a metodologia adotada, de acordo com os 
critérios de abordagem, natureza, objetivos e procedimentos. A seção 4.2 trará o fluxo 
das atividades que foram desenvolvidas nessa monografia e uma projeção das atividades 
que serão desenvolvidas no trabalho de conclusão de curso 2(TCC2). Por fim, a seção 4.3 trará 
o cronograma das atividades desenvolvidas nessa monografia e o provável cronograma para 
o desenvolvimendo do TCC2.


\subsection{Classificação da Pesquisa}

Ge acordo com \citeonline{gerhardt2009}, a metodologia é o estudo da organização, dos 
caminhos a serem percorridos, para se realizar uma pesquisa, ou um estudo, ou para se 
fazer ciência. No caso de uma pesquisa científica, ela pode ser classificada em 
vários tipos: quanto à abordagem; quanto à natureza; quanto aos objetivos e quanto aos procedimentos.

\subsubsection{Quanto à Abordagem}

A classificação quanto a abordagem pode ser dividida em: pesquisa qualitativa e quantitativa. 

A pesquisa qualitativa busca explicar o porquê das coisas e 
comumente analiza dados que não são métricos, tornando dificil 
a quantificação de valores. Ela tende a salientar os
aspectos dinâmicos, holísticos e individuais da experiência humana, para apreender
a totalidade no contexto daqueles que estão vivenciando o fenômeno \cite{gerhardt2009}.

Já a pesquisa quantitativa, tem resultados que podem ser quantificados e se baseia na 
análise de dados brutos, e tende a enfatizar o raciocínio dedutivo, as regras da lógica 
e os atributos mensuráveis da experiência humana

Tendo o conhecimento desses conceitos, nessa monografia, a pesquisa é, predominantemente, 
qualitativa, por possuir resultados não métricos que precisam de interpretação. 


\subsubsection{Quanto à Natureza}

Segundo \cite{gerhardt2009} a pesquisa quanto a natureza pode ser dividida em: pesquisa 
básica e pesquisa aplicada. A pesquisa básica gerá conhecimentos novos, sem aplicação básica 
prevista e a pesquisa aplicada gerá conhecimentos para aplicação prática, dirigidos à solução de 
problemas específico. Tendo o objetivo de desenvolver um aplicativo com os conhecimentos 
adiquiridos, essa pesquisa é de natureza aplicada.


\subsubsection{Quanto aos Objetivos}

Quanto aos objetivos, \citeonline{gil1991} classifica as pesquisas em exploratória, 
descritivas e explicativas. 

A pesquisa exploratória tem como objetivo o aprimoramento 
de ideias, ou a descoberta de intuições, o que a faz ter um planejamento bem flexivel. 
Essas pesquisas envolvem: (i) levantamento bibliográfico, (ii) entrevistas com pessoas 
que tiveram experiência prática com o problema pesquisado e (iii) análise de exemplos que 
"estimulam a compreensão".


A pesquisa descritiva tem como objetivo principal descrever caracteristicas de determinada 
população ou fenomeno, ou o estabelecimento de relaçoes entre variáveis, definindo técnicas 
padronizadas de coleta de dados, tais como, questionários e observação sistemática. A 
pesquisa explicativa têm objetivo principal identificar os fatores que determinam ou contribuem 
para a ocorrência  dos fenômenos. 

Essa monografia é portanto, predominantemente, exploratória, 
por envolver levantamento bibliográfico e questionários com mulheres que tem experiência 
prática com o problema. A própria definição do tema, teve em si, um caráter intuitivo. 

\subsubsection{Quanto aos Procedimentos}

\citeonline{gil1991}, separa a pesquisa quanto aos 
procedimentos em: experimental; bibliográfica; 
documental; de campo; Ex-Post-Facto; 
de levantamento; com survey; estudo de caso; 
participante; pesquisa-ação; etinográfica e etnometodológica.

Essa pesquisa quanto aos procedimentos pode ser classificada como: pesquisa bibliográfica, 
estudo de caso e pesquisa-ação.
Ela é bibliográfica porquê utiliza do levantamento de referências teóricas já análisadas, como livros e artigos ciêntificos. É estudo de caso por 
ser um aplicado a uma comunidade específica, criada pela própria autora, com 
mulheres que tiveram interesse no tema e é de 
pesquisa-ação porquê exise o envolvimento da autora, 
que esteve participando ativamente ao longo das 
atividades aplicadas ao grupo.


\subsection{Fluxo das Atividades}

O fluxo a seguir, foi contruído utilizando a notação BPMN e apresenta 
as atividades já desenvolvidas ao longo da execução do 
TCC1 e as que serão desenvolvidas a partir do TCC2. 
Adicionalmente, as próximas sessões detalharão de 
forma mais específica cada etapa desse processo.


\subsubsection{Definição do Tema}
Definir Tema: Esta atividade, já realizada, teve por objetivo escolher um tema,
junto aos orientadores e com base em gaps tecnológicos da área. Diante das pesquisas
realizadas, escolheu-se trabalhar com geração de código a partir de modelos orientados à
meta.
\subsubsection{Levantamento Bibliográfico}
Realizar Levantamento Bibliográfico: Esta atividade, já realizada, teve por
objetivo obter um conhecimento inicial dos autores bem como de publicações pertinentes
aos tópicos de interesse do tema dessa monografia.

\subsubsection{Proposta inicial}
Elaborar Proposta Inicial: Esta atividade, já realizada, teve por objetivo elaborar uma proposta que embasasse e justificasse a necessidade de geração de código a
partir de modelos orientados à meta. Conforme já acordado antes, optou-se por desenvolver uma aplicação que permita modelar usando a notação i* bem como gerar um código
preliminar, utilizando-se de princípios do framework MDA e orientando-se pelo padrão
arquitetural MVC.
\subsubsection{Suporte Tecnológico}
Definir Suporte Tecnológico: Esta atividade, já realizada, teve por objetivo
levantar as tecnologias para apoiar tanto o desenvolvimento da aplicação, quanto as necessidades de gerenciamento e pesquisa desse trabalho.
Definir Referencial Teórico: Esta atividade, já realizada, teve por objetivo
levantar as principais fontes conceituais para embasar o presente trabalho.
Figura 9 – Fluxo de Atividades do Trabalho
\subsubsection{Metodologia de pesquisa}
Definir Metodologia de Pesquisa: Esta atividade, já realizada, teve por objetivo definir a metodologia para guiar a pesquisa, permitindo classificar essa última quanto
à abordagem, à natureza, aos objetivos, e aos procedimentos. Adicionalmente, ao conhecer
melhor a pesquisa, e como resultado dessa atividade, foi possível estabelecer um fluxo de
atividades bem definido.
\subsubsection{Proposta da Aplicação}
Definir Proposta da Aplicação: Esta atividade, já realizada, teve por objetivo
delimitar o escopo da proposta, estabelecendo com maior detalhamento os principais
objetivos desse trabalho, em seu âmbito geral - objetivo geral - e específico - objetivos
específicos.
\subsubsection{Prova de Conceito}
Implementar prova de Conceito: Esta atividade, já realizada, teve por objetivo
implementar parte da proposta, desde a primeira parte do trabalho, permitindo avaliar
a viabilidade do projeto que se esperava realizar, em sua plenitude, na segunda parte do
trabalho. Tal atividade orientou-se pelo fluxo apresentado na seção 4.2.1.

\subsubsection{Refinar Monografia}
Refinar Monografia: Esta atividade, já realizada, teve por objetivo realizar ciclos
de revisão, junto aos orientadores, visando aprimorar a monografia.

\subsubsection{Apresentar Trabalho}
Esta atividade visou apresentar todas as atividades realizadas na primeira parte do trabalho, para os membros da banca, com o intuito de
coletar outras impressões, as quais serviram de insumos relevantes para aprimoramento
do trabalho.

\subsubsection{Realizar Correções da Banca}
: Esta atividade, já realizada, objetivou revisar o
trabalho, com base nas impressões coletadas junto aos membros da banca.

\subsubsection{Implementar Aplicação}
: Esta atividade, já realizada, teve o objetivo de finalizar a implementação da aplicação, a qual foi iniciada na prova de conceito bem como
representa o principal produto de software obtido ao final desse trabalho. Tal atividade
orientou-se pelo fluxo apresentado na seção 4.2.1.

\subsubsection{Análise de Dados Obtidos}
Análise de Dados : Esta atividade, já realizada, visou analisar a aplicação, ou seja, se essa atendeu ao esperado (lê-se proposto); além de levantar se os objetivos
foram alcançados.

\subsubsection{Refinar Monografia}
Refinar Monografia: Esta atividade, já realizada, previu realizar novos ciclos de
revisão, junto aos orientadores, visando aprimorar a monografia.

\subsubsection{Apresentar Trabalho}
Apresentar Trabalho: Esta atividade visa apresentar o resultado final do trabalho aos membros da banca; novamente, com o intuito de coletar as impressões desses
membros. Havendo necessidade, será realizada mais uma atividade para refinamento da
monografia

\subsubsection{Realizar Correções da Banca}


\subsubsection{Atividades de Desenvolvimento}

\subsubsection{Pesquisa-ação}

\subsubsection{Análise de Resultados}

\subsection{Cronograma}

\subsection{Resumo do Capítulo}

          
Do ponto de vista dos procedimentos técnicos esta é uma pesquisa 
experimental que procura estabelecer uma relação entre as causas e os efeitos de um determinado fenômino. Este fenômeno seria, os efeitos do ciclo menstrual femino na performance individual das mulheres nas tarefas cotididanas.

